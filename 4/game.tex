\section{Game Theory}
\subsection{Two-person Zero-sum Games}
Consider a game where there are two players called Player I and Player II.
Player I has $m$ choices of strategies labeled by $i\in\{1,\ldots,m\}$ and Player II has $n$ choices of strategies $j\in\{1,\ldots,n\}$.
We assume that the game is zero-sum, i.e. if Player I chooses strategy $i$ and Player II chooses strategy $j$, then Player I is paid $a_{i,j}$ and Player II is paid $-a_{i,j}$.
So the net payment across the game is zero.
The matrix $A=(a_{i,j})$ is called the payoff matrix of the game.
This is an $m\times n$ matrix.
Assume that both players want to maximise their payoff and they know the other wants it too.
\footnote{And they know the other knows they want it, and they know the other knows they know the other wants it, etc.}\\
Player I might want to solve the maximisation of $\min_ja_{ij}$ subject to $i\in\{1,\ldots,m\}$ and Player II to solve the minimisation of $\max_ia_{ij}$ subject to $j\in\{1,\ldots,n\}$.
\begin{example}
    Consider
    $$A=\begin{pmatrix}
        1&2\\
        3&4
    \end{pmatrix}$$
    Then obviously Player I will pick $i=2$ and Player II will pick $j=1$.
    This is independent of who playing first.
\end{example}
Such a point $(2,1)$ in the above example with the said property is called a saddle point of the matrix.
\begin{definition}
    A saddle point of a payoff matrix $A$ is a pair of strategies $(i,j)$ with the property
    $$a_{i,j}=\max_{i'}\min_{j'}a_{i',j'}=\min_{j'}\max_{i'}a_{i',j'}$$
    If a payoff matrix $A$ has saddle point $(i,j)$, then $a_{i,j}$ is called the value of the game.
\end{definition}
Not all payoff matrices have a saddle point, an easy example is
$$A=\begin{pmatrix}
    4&2\\
    1&3
\end{pmatrix}$$
where $\max_i\min_ja_{i,j}=2\neq 3=\min_j\max_ia_{i,j}$.
\subsection{Pure and Mixed Strategies}
Often, the game has the property that the players choose their strategies simultaneously.
The idea to analyse these games is to randomise the strategies.
\begin{definition}
    A mixed strategy is an assignment of probabilities to each of the individual strategies.
    A pure strategy is a mixed strategies assigning $1$ to one particular strategy and $0$ to all others.
\end{definition}
We use the notation that Player I plays strategy $i$ with probability $p_i$ and Player II plays strategy $j$ with probability $q_j$.
Therefore Player I would want to maximise
$$\min_j\mathbb E[\text{payout}|\text{Player II picks strategy $j$}]$$
which translates to maximising
$$\min_j\sum_{i=1}^mp_ia_{i,j}$$
subject to 
$$\sum_{i=1}^mp_i=1,\forall i\in\{1,\ldots,m\},p_i\ge 0$$
We can turn this into a linear program.
Introducing $e=(1,\ldots,1)^\top$ (dimension implied by context), then the problem can be written as maximising $v\in\mathbb R$ subject to $A^\top p\ge ve,e^\top p=1,p\ge 0$.
Similarly, the optimisation problem for Player II can be written as minimising $w$ subject to $Aq\le we, e^\top q=1,q\ge 0$.\\
Unsurprisingly, these two problems are naturally dual to each other.
Indeed, one can observe that the Lagrangian of Player I is
\begin{align*}
    L(p,v,z,q,w)&=v+q^\top(A^\top p-z-ve)+w(1-p^\top e)\\
    &=w+v(1-q^\top e)+p^\top(Aq-we)-z^\top q
\end{align*}
where $z$ is the slack variable.
So the feasible Lagrange multipliers are constrained by $q\ge 0,Aq\le we,e^\top q=1$ and $\sup L=w$.\\
By the fundamental theorem of linear programs, $(p,v)$ is optimal iff there is some $(q,w)$ such that both tuples are feasible and
$$(Ap-we)^\top p=0=q(A^\top p-ve)$$
Where we see $wp^\top e=vq^\top e$, so $v=w$ in optimal case as one would expect.
\begin{theorem}[Fundamental Theorem of Matrix Games]
    The mixed row strategy $p$ is optimal for Player I iff there exists a mixed column strategy $q$ for Player II and $v\in\mathbb R$ such that:\\
    1. $A^\top p\ge ve,e^\top p=1,p\ge 0$.\\
    2. $Aq\le ve,e^\top q=1,q\ge 0$.\\
    3. $v=p^\top Aq$.
    In this case, $(p,v)$ is optimal for the problem of Player I and $(q,v)$ is optimal for Player II.
\end{theorem}
\begin{proof}
    Immediate.
\end{proof}
The value $v$ above is called the value of the game.
Hence, for optimal mxied strategies, complementary slackness ensures that if Player II plays $j$ with positive probability, then the conditional expected payoff given Player II plays $j$ equals the value of the game.
An analogous statement works the other way around.
In other words,
$$\begin{cases}
    q_j>0\implies (A^\top p)_j=v\\
    p_i>0\implies (Aq)_i=v
\end{cases}$$
\begin{definition}
    A game is symmetric if the payoff matrix is antisymmetric.
    That is $A^\top=-A$.
\end{definition}
\begin{example}
    The rock-paper-scissors game
    $$\begin{pmatrix}
        0&-1&1\\
        1&0&-1\\
        -1&1&0
    \end{pmatrix}$$
    is symmetric.
\end{example}
\begin{theorem}
    A symmetric game has zero value.
\end{theorem}
This is very intuitive by symmetry.
We can also make use of our oven-ready fundamental theorem.
\begin{proof}
    Suppose $(p,v,q)$ satisfy the (necessary) conditions of optimality, then by antisymmetry of $A$, the tuple $(q,-v,p)$ satisfies the (sufficient) condition of optimality.
    Hence $v=-v\implies v=0$.
\end{proof}
\subsection{Finding Optimal Strategies}
If the game admits a saddle point $(i,j)$, then by definition the pure strategy $i$ is optimal for Player I and the strategy $j$ is optimal for Player II, consequently the value of the game is $a_{i,j}$.
Otherwise, we reduce the game by eliminating dominating strategies.
A row $i$ dominates $i'$ is $a_{i,j}\ge a_{i',j}$ for any $j$, so if a row $i'$ is dominated by some other row, then Player I will never play strategy $i'$.
Similarly if there is some $j,j'$ with $a_{i,j}\le a_{i,j'}$ for any $i$, then Player II will never play $j'$.
\begin{example}
    Consider the payoff matrix
    $$A=\begin{pmatrix}
        2&3&4&2\\
        3&1&1/2&4\\
        1&3&2&3
    \end{pmatrix}$$
    One can check that there is no saddle point.
    Also, the first column dominates the fourth column, so Player II will never play the fourth column.
    After eliminating that, the first row dominates the third row, so the third row is eliminated too, leaving a modified payoff matrix
    $$A'=\begin{pmatrix}
        2&3&4\\
        3&1&1/2
    \end{pmatrix}$$
    So the optimal strategy $p$ of Player I  is of the form $(p_1,1-p_1,0)^\top$.\\
    Now the region $(A')^\top p'\ge ve$ (where $p'=(p_1,1-p_1)^\top$ are the remaining relevant strategies) is given by
    $$\begin{cases}
        2p_1+3(1-p_1)\ge v\implies v\le 3-p_1\\
        3p_1+(1-p_1)\ge v\implies v\le 1+2p_1\\
        4p_1+(1-p_1)/2\ge v\implies v\le (1+7p_1)/2
    \end{cases}$$
    So by sketching the graph of $v$ against $p_1$, we find that the maximum $v$ occurs at $p_1=2/3$ and the optimal strategy of Player I is $p=(2/3,1/3,0)^\top$, and the value of the game is $v=7/3$.\\
    By complementary slackness $q_3=0$ as the third constraint is not tight.
    So $q=(q_1,1-q_1,0,0)^\top$ for some probability $q_1$.
    As $p_1=2/3>0$, the first dual constraint is binding by complementary slackness, so
    $$2q_1+3(1-q_1)=7/3\implies q_1=2/3$$
    Therefore $q=(2/3,1/3,0,0)^\top$.
\end{example}
If it does not get clear after using these tricks, one can always use the simplex algorithm.
If $\min_{i,j}a_{i,j}>0$ then we know $v$ is strictly positive, hence we can rewrite the problem of Player I as maximising $v$ subject to $A^\top x\ge e,e^\top x=1/v,x\ge 0$ where $x=p/v$.
This is equivalent to minimising $e^\top x$ subject to $A^\top x\ge e,x\ge 0$ which is a form where we can apply simplex algorithm to.
Note that its dual problem is maximising $e^\top y$ subject to $Ay\le e,y\ge 0$ which is in a form for Phase I of the two-phase algorithm.\\
If instead $\min_{i,j}a_{i,j}\le 0$, we can still use the same idea.
Just find some $k$ such that $a'_{i,j}=a_{i,j}+k>0$ for all $i,j$ and roll out the same algorithm on $A'=(a'_{i,j})$.
Then the optimal strategy of $A,A'$ would be the same and the new value $v'$ would satisfy $v=v'-k$ where $v$ is the value of the original game.
\begin{example}
    Consider the game of Rock-Paper-Scissors, i.e.
    $$A=\begin{pmatrix}
        0&-1&1\\
        1&0&-1\\
        -1&1&0
    \end{pmatrix}$$
    Then we consider the modified game
    $$A'=\begin{pmatrix}
        2&1&3\\
        3&2&1\\
        1&3&2
    \end{pmatrix}$$
    where $k=2$.
    So the problem of Player II is to maximise $y_1+y_2+y_3$ subject to
    $$\begin{cases}
        2y_1+y_2+3y_3\le 1\\
        3y_1+2y_2+y_3\le 1\\
        y_1+3y_2+2y_3\le 1\\
        y_1,y_2,y_3\ge 0
    \end{cases}$$
    So the initial simplex tableau is
    $$\begin{array}{c|cccccc|c}
        &&&&\ast&\ast&\ast&\\
        &y_1&y_2&y_3&z_1&z_2&z_3&\\ \hline
        z_1&2&1&3&1&0&0&1\\
        z_2&3&2&1&0&1&0&1\\
        z_3&1&3&2&0&0&1&1\\ \hline
        \text{Payoff}&1&1&1&0&0&0&0
    \end{array}$$
    and the final tableau would be
    $$\begin{array}{c|cccccc|c}
        &\ast&\ast&\ast&&&&\\
        &y_1&y_2&y_3&z_1&z_2&z_3&\\ \hline
        y_3&0&0&1&7/18&-5/18&1/18&1/6\\
        y_1&1&0&0&1/18&7/18&-5/18&1/6\\
        y_2&0&1&0&-5/18&1/18&7/18&1/6\\ \hline
        \text{Payoff}&0&0&0&-1/6&-1/6&-1/6&-1/2
    \end{array}$$
    Therefore $v'=2$, so $v=0$.
\end{example}