\section{Convex Optimisation}
Consider the simple problem of minimizing $f(x)$ subject to $a\le x\le b$ in the case where $f:\mathbb R\to\mathbb R$ is twice differentiable.
\begin{theorem}
    If $x^\star\notin \{a,b\}$ is an optimal solution of the problem, then $f^\prime(x^\star)=0$.
\end{theorem}
\begin{proof}
    For sufficiently small $\epsilon>0$ we can have both $x^\star-\epsilon$ and $x^\star+\epsilon$ are feasible (a fancy way of saying $(x^\star-\epsilon,x^\star+\epsilon)\subset [a,b]$), then by optimality
    $$\frac{f(x^\star)-f(x^\star-\epsilon)}{\epsilon}\le0\le\frac{f(x^\star+\epsilon)-f(x)}{\epsilon}$$
    Sending $\epsilon\to0$ gives the result.
\end{proof}
As well known, there is a partial converse to this theorem.
\begin{theorem}[Sufficient Condition for Optimality]
    Suppose $x^\star$ is feasible and $f^\prime(x^\star)=0$.
    If $f^{\prime\prime}\ge 0$ for all feasible $x$, then $x^\star$ is optimal.
\end{theorem}
\begin{proof}
    By Taylor's Theorem (used correctly!), let $x$ be a feasible solution, then
    $$f(x)=f(x^\star)+f^\prime(x^\star)(x-x^\star)+\frac{1}{2}f^{\prime\prime}(\xi)(x-x^\star)^2$$
    for some $\xi$ in between $x$ and $x^\star$.
    As $f^\prime(x^\star)=0$ and $f^{\prime\prime}(\xi)\ge 0$ by assumption, $f(x)\ge f(x^\star)$.
\end{proof}
The one-dimensional optimisation problems are widely studied and there is not much point to continue talking about it.
Let's move on to higher dimensions.
As it turns out, convexity helps a lot on problems without a functional constraint.
\subsection{General Ideas on Convexity}
Notationally, for a differentiable function $f:\mathbb R^n\to\mathbb R$, we write $Df$ as its gradient and, when $f$ is twice differentiable, $D^2f$ as its Hessian.
Recall that the Hessian is symmetric at $x$ when all the second order partial derivatives are continuous at $x$.
\begin{definition}
    A set $X\subset\mathbb R^n$ is convex if for every pair of points $x,y\in X$ and number $0<p<1$ we have $px+(1-p)y\in X$.\\
    Loosely speaking, the line segment joining $x$ and $y$ is still in $X$.
\end{definition}
One should note that our solution to the previous one dimensional optimisation problem somehow depends on the convexness of the contraint set.
\footnote{Of course, even if it is not, we can sometimes produce similar results as well}
\begin{definition}
    Let $X\subset\mathbb R^n$ be convex.
    A function $f:X\to\mathbb R$ is convex if for every pair of points $x,y\in X$ and $p\in (0,1)$,
    $$f(px+(1-p)y)\le pf(x)+(1-p)f(y)$$
\end{definition}
Respectively, a function is concave if its negative is convex.
This definition is intuitive in one dimension, but might not be that easy to illustrate in higher dimensions on first sight.
We therefore need the following reformulation of the idea.
\begin{theorem}[Supporting Hyperplane]\label{hyperplane}
    Let $X\subset\mathbb R^n$ be convex.
    The function $f:X\to\mathbb R$ is convex iff for every $x\in X$ there exists a vector $\lambda(x)\in\mathbb R^n$ such that
    $$f(y)-f(x)\ge\lambda(x)^\top(y-x)$$
    for all $y\in X$.
\end{theorem}
\begin{remark}
    If $f$ is also differentiable, then we can take $\lambda(x)=Df(x)$.
\end{remark}
One can picture it as if the hyperplane produced by $\lambda(x)$ touches and supports the high-dimensional object that is the graph of the function.
This is the origin of the name.\\
We will prove the case when $f$ is differentiable, but the general case is also true.
\begin{proof}
    Almost immediate but let's write this out.
    First suppose $\lambda(x)$ exists for all $x$.
    Fix $y,z\in X$ and $p\in (0,1)$ and let $x=py+(1-p)z$, then
    $$f(y)-f(x)\ge\lambda(x)^\top(y-x),f(z)-f(x)\ge\lambda(x)^\top(z-x)$$
    So
    $$pf(y)+(1-p)f(z)\ge\lambda(x)^\top(py+(1-p)z-x)+f(x)=f(x)$$
    Conversely, if $f$ is convex, we can just take $\lambda(x)=Df(x)$.
\end{proof}
Hence, for the problem of minimising $f(x)$ subject to $x\in X$ where $X$ is convex and $f:X\to\mathbb R$ is differentiable, we have
\begin{theorem}
    Suppose $x^\star$ is feasible and $Df(x^\star)=0$.
    If $f$ is convex, then $x^\star$ is optimal.
\end{theorem}
\begin{proof}
    Suppose $x$ is feasible, then by convexity of $f$,
    $$f(x)-f(x^\star)\ge(x-x^\star)^\top Df(x^\star)=0$$
    which is what we wanted.
\end{proof}
\begin{definition}
    A symmetric $n\times n$ matrix $A$ is non-negative definite if for every $x\in\mathbb R^n$ we have $x^\top Ax\ge 0$.
\end{definition}
\begin{theorem}
    Let $X\subset\mathbb R^n$ be convex and suppose $f:X\to\mathbb R^n$ is $C^2$.
    If $D^2f$ is nonnegative definite for all $x$, then $f$ is convex.
\end{theorem}
\begin{proof}
    For any $x,y\in X$ we have, by Taylor's Theorem that
    $$f(y)=f(x)+(y-x)^\top Df(x)+\frac{1}{2}(y-x)^\top D^2f(\xi)(y-x)$$
    where $\xi=px+(1-p)y$ for some $p\in(0,1)$.
    Therefore $f(y)-f(x)\ge Df(x)^\top(y-x)$, hence it is convex.
\end{proof}
\begin{definition}
    Given a convex set $X\subset\mathbb R^n$, a function $f:X\to\mathbb R$ is strictly convex if for every $x,y\in X$ where $x\neq y$ and every number $p\in (0,1)$ we have
    $$f(px+(1-p)y)<pf(x)+(1-p)f(y)$$
\end{definition}
\begin{definition}
    An $n\times n$ matrix $A$ is positive definite if $x^\top Ax>0$ for all $x\in\mathbb R^n\setminus\{0\}$.
\end{definition}
\begin{theorem}
    Suppose $f:X\to\mathbb R$ is twice differentiable.
    If $D^2f$ is positive definite over all of $X$, then $f$ is strictly convex.
\end{theorem}
\begin{proof}
    Same idea as in the ordinary convex case.
\end{proof}
\begin{theorem}[Uniqueness of Optimal Solutions]
    Suppose $x^\star$ and $y^\star$ are optimal solutions to the problem of minimising $f(x)$ subject to $x\in X$ with $X$ convex and $f$ strictly convex, then $x^\star=y^\star$.
\end{theorem}
\begin{proof}
    If $x^\star\neq y^\star$, then let $z=(x^\star+y^\star)/2$, then
    $$f(z)<\frac{1}{2}f(x^\star)+\frac{1}{2}f(y^\star)=f(x^\star)$$
    Contradiction.
\end{proof}
\begin{definition}
    A function $f:X\to\mathbb R$ is strongly convex if there exists a constant $m>0$ such that the function $x\mapsto f(x)-m\|x\|^2/2$ is convex.
\end{definition}
This defintion is certainly strange and unintuitive, but as we proceed, we will unveil the reason to introduce this notion.
\begin{theorem}
    Suppose $f$ is twice differentiable.
    Then $f$ is strongly convex if there exists $m>0$ such that for all $x\in X$ the matrix $D^2f(x)-mI$ is non-negative definite.
    Equivalently, $z^\top D^2f(x)z\ge m\|z\|^2$ for all $z\in\mathbb R^n$.
\end{theorem}
\begin{proof}
    Obvious.
\end{proof}
One can check that strongly convex functions are also strictly convex.
\begin{theorem}[Existence of an Optimal Solution]
    Suppose $X\subset\mathbb R^n$ is closed and $f$ is strongly convex, then there exists an optimal solution to the problem of minimising $f(x)$ subject to $x\in X$.
\end{theorem}
\begin{proof}
    Let $g(x)=f(x)-m\|x\|^2/2$ where $m>0$ such that $g$ is convex, then there is a vector $\lambda=\lambda(0)$ such that
    $$g(x)\ge g(0)+\lambda^\top x\ge g(0)-\|\lambda\|\|x\|$$
    by Cauchy-Schwartz.
    So for $\|x\|>R=2\|\lambda\|/m$ we have $f(x)>f(0)$ by rearranging.
    So our problem reduces to minimising $f(x)$ subject to $x\in X\cap B$ where $B$ is the closed ball of radius $R$.
    But $X\cap B$ is closed and bounded, hence is compact, therefore $f$ attains a minimum somewhere in it, which is the optimal sulution we want.
\end{proof}
\begin{theorem}[Gradient Lower Bound]\label{gradient_lower_bound}
    Suppose $f:X\to\mathbb R$ is differentiable and strongly convex with constant $m>0$, then
    $$\|Df(x)\|^2\ge 2m(f(x)-f(y))$$
    for any $x,y\in X$.
\end{theorem}
\begin{proof}
    From the Theorem \ref{hyperplane},
    $$f(y)-f(x)\ge(y-x)^\top Df(x)+\frac{m}{2}\|y-x\|^2$$
    However, by simply conpleting the square, for any $b,z\in\mathbb R^n$,
    $$b^\top z+\frac{m}{2}\|z\|^2\ge-\frac{\|b\|^2}{2m}$$
    Combining the inequalities give the solution.
\end{proof}
\subsection{Computing the Optimal Solution}
Our most familiar example of a convex set is $X=\mathbb R^n$.
We want to find algorithms to compute the optimal solution in this case.
Our first method of interest is called gradient descent.
Suppose $f$ is differentiable.
The rate of change of $f$ at a point $x\in\mathbb R^n$ in direction $u\in\mathbb R^n$ is
$$\lim_{t\to 0}\frac{f(x+tu)-f(x)}{t}=u^\top Df(x)$$
which attains minimum when $u$ is in the direction $-Df(x)$ by Cauchy-Schwartz.\\
So our algorithm will be as follows:
Start with an initial guess $x_0\in\mathbb R^n$ and pick a step suze $t>0$.
For every $k\ge 0$, we use the iteration
$$x_{k+1}=x_k-tDf(x_k)$$
to refine our guess.
Intuitively, this should get closer and closer to an optimal solution.
This might or might not be true, but for a sufficiently nice function $f$ we have the following theorem:
\begin{theorem}
    Suppose $f$ is $C^2$ and there exists positive constants $M>m>0$ such that $mI\preceq D^2f(x)\preceq MI$ for all $x\in\mathbb R^n$.
    Here, two symmetric matrices $A,B$ has $A\preceq B$ iff $x^\top Ax\le x^\top Bx$ for every $x\in\mathbb R^n$.
    The lower bound, in particular, implies the strong convexity, hence existence and uniqueness of $x^\star$.
    Then the gradient algorithm with step size $t=1/M$ has
    $$f(x_k)-f(x^\star)\le\left( 1-\frac{m}{M} \right)^k(f(x_0)-f(x^\star))$$
\end{theorem}
\begin{proof}
    Fix $x,y\in\mathbb R^n$, then by Taylor's Theorem, there is some $p\in (0,1)$ such that for $\xi=px+(1-p)y$ we have
    \begin{align*}
        f(y)&=f(x)+(y-x)^\top Df(x)+\frac{1}{2}(y-x)^\top D^2f(\xi)(y-x)\\
        &\le f(x)+(y-x)^\top Df(x)+\frac{M}{2}\|y-x\|^2
    \end{align*}
    Apply this inequality to $y=x_{k+1},x=x_k$ gives
    \begin{align*}
        f(x_{k+1})-f(x_k)&\le (x_{k+1}-x_k)^\top Df(x_k)+\frac{M}{2}\|x_{k+1}-x_k\|^2\\
        &=\left( -t+\frac{M}{2}t^2 \right)\|Df(x_k)\|^2\\
        &=-\frac{1}{2M}\|Df(x_k)\|^2\\
        &\le-\frac{m}{M}(f(x_k)-f(x^\star))
    \end{align*}
    Rearranging gives
    $$f(x_{k+1})-f(x_k)\le\left( 1-\frac{m}{M} \right)(f(x_k)-f(x^\star))$$
    Using this recursively gives the estimate we need.
\end{proof}
This is certainly a very nice method since the rate of convergence we arrive at looks quite promising.
There is, of course, a better method similar to our familiar formualtion of Newton's method of finding root (since very roughly speaking we are finding a root of the derivative) but in high dimensions.
To confuse people, we shall also call this Newton's method.\\
Suppose $f$ is $C^2$ and let $x_0$ be an initial guess for the optimiser, then by Taylor's Theorem,
$$f(x)\approx f(x_0)+(x-x_0)Df(x_0)+\frac{1}{2}(x-x_0)^\top D^2f(x_0)(x-x_0)$$
as $x\to x_0$ due to the continuity of $D^2f$.
Minimising the quadratic term on the right yields the approximation
$$x^\star\approx x_0-(D^2f(x_0))^{-1}Df(x_0)$$
Formally, we are referring to the recursion
$$x_{k+1}=x_k-(D^2f(x_0))^{-1}Df(x_0)$$
As before, we want to know if it works.
\begin{definition}
    Let $A$ be an $n\times n$ matrix.
    We define the matrix norm $\|A\|$ as
    $$\|A\|=\inf\{a\ge 0:\forall z\in\mathbb R^n,\|Az\|\le a\|z\|\}=\sup_{\|z\|=1}\|Az\|$$
    In particular, if $A$ is nonnegative definite, then $\|A\|$ is the smallest eigenvalue of $A$.
\end{definition}
\begin{theorem}
    Suppose $f$ is twice differentiable and there exists constants $m,L>0$ with $D^2f(x)\succeq mI$ (so $f$ is strongly convex) and
    $$\forall x,y\in\mathbb R^n,\|D^2f(x)-D^2f(y)\|\le L\|x-y\|$$
    (so $D^2f$ is $L$-Lipschitz), then Newton's method guarantees that
    $$f(x_k)-f(x^\star)\le \frac{2m^3}{L^2}\left(\frac{L}{2m^2}\|Df(x_0)\|\right)^{2^{k+1}}$$
\end{theorem}
Note that this convergence rate is way quicker than using gradient descent.
\begin{proof}
    Denote $\Delta x_k=x_{k+1}-x_k=-(D^2f(x_0))^{-1}Df(x_0)$, then we have
    \begin{align*}
        \|Df(x_{k+1})\|&=\|Df(x_{k+1})-Df(x_k)-D^2f(x_k)\Delta x_k\|\\
        &=\left\|\int_0^1(D^2f(x_k+t\Delta x_k)-D^2(x_k))\Delta x_k\,\mathrm dt\right\|\\
        &\le\int_0^1\|(D^2f(x_k+t\Delta x_k)-D^2(x_k))\Delta x_k\|\,\mathrm dt\\
        &\le L\|\Delta x_k\|^2\int_0^1t\,\mathrm dt\\
        &=\frac{1}{2}L\|(D^2f(x_0))^{-1}Df(x_0)\|^2\\
        &\le\frac{L}{2m^2}\|Df(x_k)\|^2
    \end{align*}
    Note here that we have used the fact that $A\succeq mI\implies \|A^{-1}\|\le 1/m$ which is quite obvious.
    Applying this inequality recursively yields
    $$\sqrt{2m(f(x_k)-f(x^\star))}\le\|Df(x_k)\|\le\frac{2m^2}{L}\left( \frac{L}{2m^2}\|Df(x_0)\| \right)^{2^k}$$
    by Theorem \ref{gradient_lower_bound}.
    Rearranging shows what we want.
\end{proof}