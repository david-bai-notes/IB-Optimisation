\section{Introduction}
Our aim for this course is to explore optimisation problems, typically minimising a function $f(x)$ with $f:\mathbb R^n\to\mathbb R$ the objective function subject to the regional constraint $g(x)=b,x\in X$ where $X\subset\mathbb R^n$ defines a regional constraint, and $g:\mathbb R^n\to\mathbb R^m$ defines $m$ functional constraints.\\
A feasible solution to the problem is any $x\in X$ such that $g(x)=b$, so a feasible solution is just one that satisfy the constraint of the problem.
An optimal solution is a feasible solution $x^\star$ such that $f(x^\star)\le f(x)$ for all feasible $x$.
The problem itself is feasible if there exists at least one feasible solution, and it is bounded if $\inf\{f(x):g(x)=b,x\in X\}\in\mathbb R$, which is the main focus of us.\\
Notice that it makes no difference if we consider the maximisation instead of minimisation by putting appropriate minus signs in the definitions.