\section{The Lagrangian}
\subsection{The Sufficiency Theorem}
We now consider the general case of the constrained optimisation, without explicitly appealing to the convexity assumption.
Suppose we want to minimise $f(x)$ subject to $g(x)=b,x\in X$, to deal with the functional constraint, we introduce the Lagrangian of the problem:
\begin{definition}[Lagrangian]
    The Lagrangian $L:\mathbb R^n\times\mathbb R^m\to\mathbb R$ of the problem is defined by
    $$L(x,\lambda)=f(x)+\lambda^\top(b-g(x))$$
    For $\lambda=(\lambda_1,\ldots,\lambda_m)^\top$, the components $\lambda_i$ constitutes the Lagrange multiplier.
\end{definition}
A nice theorem associated with this notion is the following:
\begin{theorem}[The Lagrange Sufficiency Theorem]\label{lagrange_suff}
    Let $x^\star$ be feasible for the problem.
    Suppose there exists a $\lambda^\star\in\mathbb R^m$ such that
    $$\forall x\in X,L(x^\star,\lambda^\star)\le L(x,\lambda^\star)$$
    Then $x^\star$ is optimal.
\end{theorem}
\begin{proof}
    For any feasible $x$ and any $\lambda\in\mathbb R^m$, we have
    $$L(x,\lambda)=f(x)+\lambda^\top(b-g(x))=f(x)$$
    So our assumption gives, for any feasible $x$,
    $$f(x^\star)=L(x^\star,\lambda^\star)\le L(x,\lambda^\star)=f(x)$$
    as desired.
\end{proof}
\begin{example}
    Consider the problem of minimising $x_1^2+3x_2^2$ subject to $4x_1+x_2=7$.
    We claim that $(12/7,1/7)$ is optimal.
    The Lagrangian has the expression
    $$L(x_1,x_2,\lambda)=x_1^2+3x_2^2+\lambda(7-4x_1-x_2)$$
    Take $\lambda=6/7$, we have
    $$L(x_1,x_2,6/7)=(x_1-12/7)^2+3(x_2-1/7)^2+3\ge 3=L(12/7,1/7,6/7)$$
    So $(12/7,1/7)$ is optimal by the preceding theorem.
\end{example}
Note that our choice of multiplier $\lambda^\star=6/7$ is kind of ``out of nowhere''.
It makes one wonder if there is a systematic method to guarantee to produce a $\lambda^\star$ that works.
Sadly, it is not always possible to find a multiplier that works as a certificate of optimality, let alone a sufficient algorithm to compute one.
However, inspired by the theorem, we can come up with a way to compute $\lambda^\star$ if the problem behaves sufficiently nice.\\
If $x^\star$ and $\lambda^\star$ do exist, we must have
$$\inf_{x\in X}L(x,\lambda^\star)=f(x^\star)>-\infty$$
So to calculate $\lambda^\star$, we first identify the possible candidates
$$\Lambda=\{\lambda\in\mathbb R^n:\inf_{x\in X}L(x,\lambda)>-\infty\}$$
Then, for each $\lambda\in\Lambda$, we consider the probelm of minimising $L(x,\lambda)$ subject to $x\in X$.
This is hopefully easiler since the functional constraint has been removed (with, of course, the trade-off of having to consider the problem for a wide range of functions).
Denote then minimiser as $x(\lambda)$.
We now find a $\lambda^\star\in\Lambda$ such that $x(\lambda^\star)$ is feasible.
In general, it might not be possible to go through all these steps, but if it is possible, then $x^\star=x(\lambda^\star)$ would be guaranteed to be the optimal solution by the theorem.
\begin{example}[Maximum Likelihood Estimator of the Multinomial Distribution]
    Given constants $n_1,\ldots,n_k>0$, we consider the problem to maximise $\sum_in_i\log p_i$ subject to $\sum_ip_i=1,\forall i,p_i>0$.
    The Lagrangian is
    $$L(p,\lambda)=\lambda+\sum_{i=1}^k(n_i\log p_i-\lambda p_i)$$
    Then easily $\Lambda=\mathbb R_{>0}$.
    Now we have
    $$\frac{\partial L}{\partial p_i}=\frac{n_i}{p_i}-\lambda=0\implies p_i(\lambda)=\frac{n_i}{\lambda}$$
    But also
    $$D^2L=\begin{pmatrix}
        -n_1/p_1^2&0&0\\
        0&\ddots&0\\
        0&0&-n_k/p_k^2
    \end{pmatrix}$$
    which is negative definite, so we have found the maximum.
    The constraint $\sum_ip_i=1$ gives $\lambda=\sum_in_i$, so
    $$p_i^\star=\frac{n_i}{\sum_jn_j}$$
    is optimal.
\end{example}
\subsection{Inequality Constraints}
We can consider a somewhat more general version of the optimisation problem, where we replace the functional constraints by their inequality analog.
For vectors $x,y$, we write $x\le y$ if $x_i\le y_i$ for each $i$.\\
Let $f:\mathbb R^n\to\mathbb R,g:\mathbb R^n\to\mathbb R^m,X\in\mathbb R^n$.
We want to consider the problems to minimise $f(x)$ subject to $g(x)\le b,x\in X$.
Notice that we can actually rewrite this problem to an equality form by introducing a slack variable $z\in\mathbb R^m$ by considering the minimisation of $f(x)$ subject to $g(x)+z=b,x\in X,z\ge 0$.\\
Then, the Lagrangian of the modified problem would be
$$L(x,z,\lambda)=f(x)+\lambda^\top(b-g(x)-z)=f(x)+\lambda^\top(b-g(x))-\lambda^\top z$$
Note that if some of $\lambda$ is positive then we can easily send $L$ to $-\infty$, so $\Lambda\subset\{\lambda\in\mathbb R^m:\lambda\le 0\}$.
So if we want to use the Lagrange multiplier method, then a sign constraint is further imposed on $\lambda$.
In particular,
$$\Lambda=\{\lambda\in\mathbb R^m:\lambda\le 0,\inf_{x\in X}(f(x)+\lambda^\top(b-g(x)))>-\infty\}$$
Now we proceed to the second step of using Lagrange multiplier.
Note that for $\lambda\le 0$, we have $\inf_{z\ge 0}(-\lambda^\top z)=0$, so the optimal $z=z(\lambda)$ would satisfy the complementary slackness condition, i.e. $\lambda^\top z=0$.
So if the $i^{th}$ Lagrange multiplier is nonzero, then $z_i=0$, so the $i^{th}$ functional constraint is tight, i.e. hold with equality.
If the $i^{th}$ functional constraint is not tight, then $z_i>0$, thus it must be the case that $\lambda_i=0$.
The rest of the method can then be carried on as usual.
\begin{example}
    Consider the minimisation of $x_1-3x_2$ subject to $x_1^2+x_2^2\le 4$ and $x_1+x_2\le 2$.
    We introduce the slack variables $z_1,z_2\ge 0$, by which we can transform the problem to the minimisation of $x_1-3x_2$ subject to
    $$\begin{cases}
        x_1^2+x_2^2+z_1=4\\
        x_1+x_2+z_2=2\\
        z_1,z_2\ge 0
    \end{cases}$$
    So the Lagrangian is
    $$L=x_1-3x_3+\lambda_1(4-x_1^2-x_2^2-z_1)+\lambda_2(2-x_1-x_2-z_2),\lambda_1,\lambda_2\le 0$$
    Note that
    $$D^2L=\begin{pmatrix}
        -2\lambda_1&0\\
        0&-2\lambda_1
    \end{pmatrix}\succeq 0$$
    Hence we only need $\partial L/\partial x_1=\partial L/\partial x_2=0$, whcih yields
    $$\begin{cases}
        1-2\lambda_1x_1-\lambda_2=0\\
        -3-2\lambda_1x_2-\lambda_2=0
    \end{cases}$$
    If $\lambda_1=0$, then $\lambda_2=1$ and $\lambda_2=-3$, contradiction.
    So $\lambda_1<0$.
    if $\lambda_2<0$, then by complementary slackness, $z=0$, therefore
    $$\begin{cases}
        1-2\lambda_1x_1-\lambda_2=0\\
        -3-2\lambda_1x_2-\lambda_2=0\\
        x_1^2+x_2^2=4\\
        x_1+x_2=2
    \end{cases}$$
    Solving this equation gives $(x_1,x_2)\in\{(2,0),(0,2)\}$, so $(x_1,x_2,\lambda_1,\lambda_2)$ is either $(2,0,1,-3)$ or $(0,2,-1,1)$, but both violates the sign constraint, which gives contradiction again.
    Hence $\lambda_2=0$.
    By complementary slackness $z_1=0$, so the first functional constraint is tight, hence
    $$\begin{cases}
        1-2\lambda_1x_1=0\\
        -3-2\lambda_1x_2=0\\
        x_1^2+x_2^2=4
    \end{cases}$$
    Solve to get $\lambda_1=-\sqrt{10}/4$ and
    $$(x_1,x_2)=\left( -\sqrt{\frac{2}{5}},3\sqrt{\frac{2}{5}} \right)$$
    which is feasible hence optimal.
\end{example}
While we can solve it, it is always good to know beforehand that this method will work.
This brings us to the next topic on the necessity condition.
\subsection{The Necessity Theorem}
Let's go back to the problem of minimising $f(x)$ subject to $g(x)=b,x\in X$.
Denote the Lagrangian by
$$L(x,\lambda)=f(x)+\lambda^\top(b-g(x))$$
We have already established Theorem \ref{lagrange_suff} saying that a feasible $x^\star$ is optimal if there exists some $\lambda^\star$ with $L(x^\star,\lambda^\star)\le L(x,\lambda^\star)$ for any $x\in X$.
In particular, we have
$$\inf_{x\in X,g(x)=b}f(x)=f(x^\star)=\inf_{x\in X}L(x,\lambda^\star)$$
We, of course, want some useful statement on the converse of this statement.
\begin{theorem}[Lagrangian Necessity]\label{lagrange_nece}
    For $b\in\mathbb R^n$, let
    $$\phi(b)=\inf_{x\in X,g(x)=b}f(x)$$
    If $\phi$ is defined and convex, then there is a $\lambda=\lambda(b)$ with
    $$\phi(b)=\inf_{x\in X,g(x)=b}f(x)=\inf_{x\in X}L(x,\lambda(b))$$
    Furthermore, if $\phi$ is differentiable, then $\lambda(b)$ can be taken to be $D\phi(b)$.
\end{theorem}
\begin{definition}
    Such a function $\phi(b)$ is called the value function of the family of the problems of minimising $f(x)$ subject to $g(x)=b,x\in X$.
\end{definition}
\begin{proof}
    Suppose $\phi$ is convex and fix $b\in\mathbb R^m$, then by Theorem \ref{hyperplane}, there is some $\lambda\in\mathbb R^m$ (which can be taken to be $D\phi(b)$) such that
    \begin{align*}
        \phi(b)&=\inf_{c\in\mathbb R^m}(\phi(c)+\lambda^\top(b-c))\\
        &=\inf_{c\in\mathbb R^m}\inf_{x\in X,g(x)=c}(f(x)+\lambda^\top(b-c))\\
        &=\inf_{c\in\mathbb R^m}\inf_{x\in X,g(x)=c}(f(x)+\lambda^\top(b-g(x)))\\
        &=\inf_{x\in X}L(x,\lambda)
    \end{align*}
    which is just what we wanted.
\end{proof}
\begin{remark}
    By adding slack variables
    $$\inf_{x\in X,g(x)\le b}f(x)=\inf_{x\in X,z\ge 0,g(x)+z=b}f(x)$$
    We can see that the theorem can still hold when the functional constraint is an equality.
\end{remark}
To apply this theorem, we can require helps on determining the convexity of $\phi$.
\begin{theorem}
    Suppose $X$ is convex, $f$ is convex, the functional constraint is $g(x)\le b$ and $g_j$ is convex for all $1\le j\le m$, then $\phi$ is convex.
\end{theorem}
\begin{proof}
    Exercise.
\end{proof}
\subsection{Shadow Prices}
We can view the optimisation problem in some real-life economical setting, which gives intuition to the construction of the Lagrangian.\\
Consider a factory owner who makes $n$ different products out of $m$ raw materials.
He needs to choose amount $x_i$ to make of the $i^{th}$ product for each $i=1,\ldots,n$.
Given a vector of amounts $x=(x_1,\ldots,x_n)^\top$ of products to manufacture, the factory requires the amount $g_j(x)$ of the $j^{th}$ raw material for $j=1,\ldots,m$, and the amount of available $j^{th}$ raw material available is $b_j$.
The only nonnegative amounts of products can be produced and given the amounts $x$ of products, the profit earned is $f(x)$.\\
Obviously, what the factory owner wants is to maximise $f(x)$ subject to $g(x)\le b,x\ge 0$.
We let $\phi(b)$ be the maximised profit as a function of the amount of raw materials.\\
To solve this problem, we can consider a fictional world where there is a market for the raw material with the unit price of the $j^{th}$ raw material being $\lambda_j$, then the total profit is exactly
$$L(x,\lambda)=f(x)+\lambda^\top(b-g(x))$$
So the problem has transformed into the maximisation problem of the Lagrangian.
For each price vector $\lambda$, the optimal amount of products is the vector $x(\lambda)$.
We can now find a price vector $\lambda^\star$ such that $x^\star=x(\lambda)$ is feasible.
Such a $\lambda^\star$ is called the shadow price vector of the raw materials.
This is the prices of the raw materials such that the maximisation problem makes no difference on whether there is such a fictional market.\\
The supporting hyperplane can also have an economical interpretation.
Suppose the fictional market of raw materials suddenly appears, then the factory owner would buy the basket $(\epsilon_1,\ldots,\epsilon_m)^\top$ of raw materials if $\phi(b+\epsilon)-\phi(b)\ge\lambda^\top\epsilon$, so when $\epsilon$ is small,
$$\phi(b+\epsilon)-\phi(b)=D\phi(b)^\top\epsilon$$
So if $\lambda_j>\partial\phi/\partial b_j$, then the factory owner should not be buying any of that raw material.
Hence the shadow price has to be $D\phi(b)$ as before.\\
To check that all these are consistent, we check the followings:
The inequality constraint $g(x)\le b$ gives a sign constraint $\lambda\ge 0$, which is consistent with the assumption that price should not be negative.
Also, if the $j^{th}$ constraint is not tight, then $g_j(x^\star)<b_j$, then the owner does not use up all the $j^{th}$ raw material, so the shadow price has $\partial\phi/\partial b_j=0$ as there is no extra profit in acquiring some more of that material.
On the other hand, the $j^{th}$ slack variable would be positive, hence $\lambda_j=0$ by complementary slackness, which makes sense economically.
\subsection{Duality}
We can introduce the notion of a dual optimisation problem by a corresponding economic motivation.
Consider the point of view of the seller of the raw materials who then buy the products produced by the factory.
If the amount of finished products is $x$ and the price of raw materials is $\lambda$, then the seller's profit would be the negative of the profit of factory owner, which is $f(x)+\lambda^\top(b-g(x))$.
So what the seller wants would be to minimise
$$\sup_{x\ge 0}(f(x)+\lambda^\top(b-g(x)))$$
\begin{definition}
    Consider the primal problem of minimising $f(x)$ subject to $g(x)=b,x\in X$ which has Lagrangian
    $$L(x,\lambda)=f(x)+\lambda^\top(b-g(x))$$
    Then the set of feasible Lagrange multipliers is
    $$\Lambda=\{\lambda\in\mathbb R^m:\inf_{x\in X}L(x,\lambda)>-\infty\}$$
    Then the dual objective function is defined as
    $$h:\Lambda\to\mathbb R,h(\lambda)=\inf_{x\in X}L(x,\lambda)$$
    The dual problem of the primal problem is to maximise $h(\lambda)$ subject to $\lambda\in\Lambda$.
\end{definition}
\begin{remark}
    One should note that if one need to study the dual problem of a problem with inequality constraints, then we need to introduce the slack variables first.
    Also, we can formulate the dual problem to a maximisation problem analogously.
\end{remark}
\begin{theorem}[Weak Duality]\label{weak_dual}
    With the definitions introduced above,
    $$\sup_{\lambda\in\Lambda} h(\lambda)\le\inf_{x\in X,g(x)=b}f(x)$$
\end{theorem}
\begin{proof}
    For any $x$ feasible for the primal problem and $\lambda$ feasible for the dual problem, then
    $$h(\lambda)=\inf\{L(\xi,\lambda):\xi\in X\}\le L(x,\lambda)=f(x)$$
    As $L(x,\lambda)=f(x)$ whenever $x$ is feasible.
\end{proof}
So we can restate Theorem \ref{lagrange_suff} in the following way instead:
If $x^\star$ and $\lambda^\star$ are feasible for the primal and dual problems respectively, and $h(\lambda^\star)=f(x^\star)$, then $x^\star$ is optimal.
\begin{remark}
    The difference
    $$\inf_{x\in X,g(x)=b}f(x)-\sup_{\lambda\in\Lambda}h(\lambda)$$
    is called the duality gap.
    It is in general nonnegative by Theorem \ref{weak_dual}.
    By Theorem \ref{lagrange_nece}, if the value function of the primal problem is convex, then there is some $\lambda^\star\in\Lambda$ such that
    $$\inf_{x\in X,g(x)=b}f(x)=\inf_{x\in X}L(x,\lambda^\star)=h(\lambda^\star)$$
    So the duality gap is zero.
    This is called \textit{strong duality}.
\end{remark}
\subsection{Duality in Linear Programs}
\begin{definition}
    Linear Programs are problems of maximising $c^\top x$ subject to $Ax\le b,x\ge 0$ where $A$ is an $m\times n$ matrix, $b\in\mathbb R^m$ and $c\in\mathbb R^n$.
\end{definition}
Of course, we can analyze the dual problem too.
Introduce the slack variable $z\in\mathbb R^m$ and transform the problem to maximising $c^\top x$ subject to $Ax+z=b,x\ge 0,z\ge 0$.
So the Lagrangian is
$$L(x,z,\lambda)=c^\top x-\lambda^\top(b-Ax-z)=b^\top\lambda +(c-A^\top\lambda)^\top x-\lambda^\top z$$
So
$$\Lambda=\{\lambda\in\mathbb R^m:\sup_{x,z\ge 0}L(x,z,\lambda)<\infty\}=\{\lambda\in\mathbb R^m:A^\top\lambda\ge c,\lambda\ge 0\}$$
Now the dual objective function is
$$\sup_{x\ge 0,z\ge 0}L(x,z,\lambda)=b^\top\lambda$$
for $\lambda\in\Lambda$.
Hence the dual problem is to minimise $b^\top\lambda$ subject to $A^\top\lambda\ge c,\lambda\ge 0$.
This looks just like our original problem.
\subsection{The Barrier Method}
Consider the optimisation problem of minimising a differentiable convex function $f:\mathbb R^n\to\mathbb R$ subject to $g(x)\le b$, where $g:\mathbb R^n\to\mathbb R^m$ has every one of its components differentiable.
To tackle it (computationally), we consider the family of unconstrained minimisation problems of
$$f(x)-\epsilon\sum_{i=1}^m\log(b_i-g_i(x))$$
for $\epsilon>0$.
Note that the constraint $g(x)<b$ is implicitly imposed due to the domain of $\log$.
The idea of it, as one might have already observed, is to add an extra term which blows up to positive infinity if the condition $g(x)<b$ is not satisfied.
The reason why we use a term in the $\epsilon\log$ form is to smoothen the blow-up term for computational issues.
\begin{theorem}\label{barrier_method}
    Suppose $x^\star$ is optimal for our original problem and $x_\epsilon$ is optimal for the problem shown above, then
    $$0\le f(x_\epsilon)-f(x^\star)\le m\epsilon$$
\end{theorem}
To prove it, we shall introduce some preliminary notion.
The Lagrangian for the original problem is $L(x,z,\lambda)=f(x)+\lambda^\top (b-z-g(x))$ for the slack variable $z\ge0$ and the set of feasible Lagrange multipliers are
$$\Lambda=\{\lambda\in\mathbb R^m:\inf_{x\in\mathbb R^n,z\ge 0}L(x,z,\lambda)>-\infty\}$$
\begin{claim}
    Let $\lambda\in\mathbb R^m$ be such that $\lambda\le 0$ and that there exists some $x_\lambda\in\mathbb R^n$ with
    $$Df(x_\lambda)=\sum_{i=1}^m\lambda_iDg_i(x_\lambda)$$s
    Then $\lambda\in\Lambda$ and the dual objective function for this problem is just
    $$h(\lambda)=b^\top\lambda+f(x_\lambda)-\lambda^\top g(x_\lambda)$$
\end{claim}
\begin{proof}
    Recall that every convex differentiable function is minimised at any of its stationary points.
    Also note that $f(x)-\lambda^\top g(x)$ is convex.
    So for $x\in\mathbb R^n$ and $z\ge 0$ we have the bound
    $$L(x,z,\lambda)=b^\top\lambda+f(x)-\lambda^\top g(x)-\lambda^\top z\ge b^\top\lambda+f(x_\lambda)-\lambda^\top g(x_\lambda)$$
    Equality holds when $x=x_\lambda$ and $(z,\lambda)$ satisfies complementary slackness.
    The rest follows.
\end{proof}
\begin{proof}[Proof of Theorem \ref{barrier_method}]
    Recall that a differentiable function on an open set has derivative $0$ on any of its local minima.
    As $x_\epsilon$ is optimal to the modified problem, we necessarily have
    $$Df(x_\epsilon)=-\epsilon\sum_{i=1}^m\frac{Dg_i(x_\epsilon)}{b_i-g(x_\epsilon)}$$
    So we set, for $i=1,\ldots,m$ we define $\lambda_i=-\epsilon/(b_i-g_i(x_\epsilon))$.
    Then
    $$Df(x_\epsilon)=\sum_{i=1}^m\lambda_iDg_i(x_\epsilon)$$
    So by the preceding claim $\lambda\in\Lambda$.
    Therefore, by Theorem \ref{weak_dual},
    $$f(x_\epsilon)\ge f(x^\star)\ge h(\lambda)=f(x_\epsilon)+\lambda^\top(b-g(x_\epsilon))=f(x_\epsilon)-m\epsilon$$
    which is the bound we wanted.
\end{proof}
\begin{remark}
    As $f$ and $g_i$ are convex for all $i$, the value function $\phi(b)=\inf\{f(x):g(x)\le b\}$ is convex.
    So by fixing $b$ and assuming there exists an optimiser $x^\star$, Theorem \ref{lagrange_nece} then tells us there is some $\lambda^\star$ with
    $$L(x^\star,z^\star)=\inf\{L(x,z,\lambda^\star):x\in\mathbb R^n,z\ge 0\},z^\star=b-g(x^\star)$$
\end{remark}
Theorem \ref{barrier_method} inspires us to invent an algorithm to approach the optimal solution.
Pick an initial guess $x_0\in\mathbb R^n$ with $g(x_0)<b$ and some initial $\epsilon_0>0$.
For $k\ge 0$, we solve the modified problem with $\epsilon=\epsilon_k$ approximately (using e.g. gradient descent or Newton's method) starting from $x_k$.
Then $x_{k+1}$ is defined to be the approximated optimal solution in this way, and reduce $\epsilon$ by $\epsilon_{k+1}=r\epsilon_k$ for some $r\in (0,1)$.