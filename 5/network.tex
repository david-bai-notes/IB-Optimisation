\section{Network Flow Problems}
\subsection{The Maximal Flow Problem}
\begin{definition}
    A network (or directed graph) is a set $N=\{1,\ldots,n\}$ of nodes (or vertices) together with a collection $A\subset N\times N$ of ordered pairs $(i,j)$ called arcs or directed edges.
    The flow in the network is a collection of numbers $(x_{i,j})$ indexed by elements in $N\times N$.
\end{definition}
The problem we wish to consider is formulated in the following way:
Each arc $(i,j)$ is assigned a capacity $c_{i,j}\ge 0$.
In reality (or for notational convenience), we will pretend all arcs in $N\times N$ exists and $c_{i,j}=0$ for $(i,j)\notin A$.\\
We distinguish two special nodes in the network.
Node $1$ is called the source node while the node $n$ is called the sink.
The maximum flow problem is to pump as much flow through the network, from the source to the sink, subject to the capacity constraints and the constraint that the flow-in and flow-out cancelled out in any notes except the source and the sink.
\footnote{Normally, there is no flow-in at the source and no flow-out at the sink.}
Formally, we want to maximise $v\in\mathbb R$ subject to $0\le x_{i,j}\le c_{i,j}$ and
$$\sum_{j\in N}x_{i,j}-\sum_{j\in N}x_{j,i}=\begin{cases}
    v\text{, if $i=1$}\\
    0\text{, if $1<i<n$}\\
    -v\text{, if $i=n$}
\end{cases}$$
which, as one can discover joyfully, is a linear program.
As an aside, the variable $v$ does not have to appear as it can be expressed in terms of $(x_{i,j})$ in the constraint.
It is simply for notational convenience.\\
Of course, we can apply the barrier method or simplex algorithm and call it a day, but certainly there are more useful structures in this system that allows us to simplify it.
\subsection{The Ford-Fulkerson Algorithm}
A nice algorithm developed by Ford-Fulkerson is as follows:\\
First, assign an initial feasible flow $(x_{i,j})$ with value $v$ (for example we can take all of them to be zero).
Then find an augmented path, defined a a collection of nodes $1=i_1,i_2,\ldots,i_{r-1},i_r=n$ with eiher $c_{i_k,i_{k+1}}-x_{i_k,i_{k+1}}>0$ or $x_{i_{k+1},i_k}>0$ for each $k$.
If an augmented path exists, then increase the flow along the path by the amount
$$\delta=\min_{k\in\{1,\ldots,r-1\}}\max\{c_{i_k,i_{k+1}}-x_{i_k,i_{k+1}},x_{i_{k+1},i_k}\}$$
which is positive by assumption and
$$x_{i,j}'=\begin{cases}
    x_{i,j}+\delta\text{, if $i,j$ are consecutive in the path}\\
    x_{i,j}-\delta\text{, if $j,i$ are consecutive in the path}\\
    x_{i,j}\text{, otherwise}
\end{cases}$$
Then, as one can check, $(x_{i,j}')$ is also feasible with $v'=v+\delta$.
We shall show that the non-existence of an augmented path is equivalent to the maximality of the flow.
\begin{definition}
    Let $S\subset N$ be a collection of nodes.
    The complement of $S$ in $N$ is denoted $\bar{S}=N\setminus S$.
    A cut seperating the source and the sink is the set of arcs $(i,j)\in S\times \bar{S}$ where $\{1,n\}\in S$.
    THe capacity of a cut $(S,\bar{S})$ is defined as
    $$c(S,\bar{S})=\sum_{i\in S,j\in\bar{S}}c_{i,j}$$
\end{definition}
\begin{theorem}[The max-flow min-cut Theorem]\label{max-flow-min-cut}
    For a network, the maximum value of a feasible flow from the source to the sink equals to the minimum of all cut capacities of all cuts seperating the source and the sink.
\end{theorem}
\begin{proof}
    Fix any feasible flow $(x_{i,j})$.
    For any $S,T\subset N$ we write
    $$f(S,T)=\sum_{i\in S,j\in T}x_{i,j}$$
    be the total flow over arcs joining $S$ and $T$.
    Now let $S,\bar{S}$ be a cut seperating the source and the sink, the constraint tells us
    \begin{align*}
        v&=\sum_{i\in S,j\in N}x_{i,j}-\sum_{j\in N,i\in S}x_{ji}\\
        &=f(S,N)-f(N,S)\\
        &=(f(S,S)+f(S,\bar{S}))-(f(S,S)+f(\bar{S},S))\\
        &=f(S,\bar{S})-f(\bar{S},S)\\
        &\le f(S,\bar{S})\\
        &\le c(S,\bar{S})
    \end{align*}
    Hence the value of any flow is less than or equal to the capacity of any cut.
    To show that they are actual equal, notice that the inequality above becomes an equality iff $x_{i,j}=c_{i,j}$ and $x_{j,i}=0$ for all arcs $(i,j)$ in $(S,\bar{S})$.
    SO let $(x_{i,j}^\star)$ be a maximal flow (which exists by compactness).
    We define a set $S$ by induction in the following criteria:\\
    1. The source is in $S$.\\
    2. If $i\in S$ and $x_{i,j}^\star<c_{i,j}$, then $j$ in $S$.\\
    3. If $i\in S$ and $x_{j,i}^\star>0$, then $j\in S$.\\
    Now the sink is in $\bar S$ since otherwise there would be a path (by the inductive process) from the source to the sink pumping strictly more flow, contradicting the maximality of $(x_{i,j}^\star)$.
    The cut $(S,\bar{S})$ has the property we wanted.
\end{proof}
The same constructive idea shows that if no augmented path can be found, then the flow is maximal as one can construct such a cut which has capacity equal to the value of the flow.\\
Also, the algorithm is guaranteed to terminate if the capacities and elements of the initial flow are all integer (hence it also terminates if all are rational by multiplying an integer).
This is because the $\delta$ we obtain from the algorithm is at least $1$, so if the value of $v$ increases indefinitely through the algorithm, it cannot be bounded, which contradicts the compactness of the domain of the problem.
But if something is irrational, the algorithm may not terminate.
\subsection{The Dual Problem}
The Lagrangian of the maximal flow problem is (note that $v$ can be expressed in terms of $(x_{i,j})$):
\begin{align*}
    L&=v+\sum_i\lambda_i\left( \sum_jx_{i,j}-\sum_jx_{j,i} \right)-\lambda_1v+\lambda_nv+\sum_{i,j}\mu_{i,j}(c_{i,j}-x_{i,j}-z_{i,j})\\
    &=\sum_{i,j}c_{i,j}\mu_{i,j}+v(1-\lambda_1+\lambda_n)+\sum_{i,j}x_{i,j}(\lambda_i-\lambda_j-\mu_{i,j})-\sum_{i,j}\mu_{i,j}z_{i,j}
\end{align*}
where $z$ is the slack variable.
For $L$ to be bounded above varying $x,z$, we have
$$\begin{cases}
    1-\lambda_1+\lambda_n=0\\
    \lambda_i-\lambda_j\le\mu_{i,j}\\
    \mu_{i,j}\ge 0
\end{cases}$$
So the dual problem is to minimise $\sum_{i,j}c_{i,j}\mu_{i,j}$ subject to $\lambda_1-\lambda_n=1,\lambda_i-\lambda_j\le\mu_{i,j},\mu_{i,j}\ge 0$.
By complementary slackness, if $x$ is optimal for the primal problem and $(\lambda,\mu)$ is optimal for the dual problem, then $\lambda_i-\lambda_j=\mu_{i,j}$ and $\mu_{i,j}=0$.\\
By Theorem \ref{max-flow-min-cut}, there is a set $S$ with $1\in S,n\notin S$ such that the optimal dual variables are given by
$$\lambda_i=\begin{cases}
    1\text{, if $i\in S$}\\
    0\text{, if $i\in\bar{S}$}
\end{cases},\mu_{i,j}=\begin{cases}
    1\text{, if $i\in S,j\in\bar{S}$}\\
    0\text{, otherwise}
\end{cases}$$
due to our previously discussed dual constraints.
\subsection{Combinatorial Applications}
There are certain combinatorial applications of Theorem \ref{max-flow-min-cut}.
One combinatorial problems is formulated like this:
Suppose that a standard deck of $52$ playing cards dealt into $13$ piles of $4$ cards each, we want to show that it is possible to select exactly one card from each pile such that the selected cards contain one card of each of the $13$ ranks A,2,\dots,10,J,Q,K.\\
To prove this, we build a network with nodes
$$A,a_1,\ldots,a_{13},b_1,\ldots,b_{52},c_1,\ldots,c_{13},B$$
as follows:
The nodes $(a_i)$ represent the $13$ piles, and the nodes $(b_j)$ represent the $52$ cards, and the node $(c_k)$ represent the $13$ ranks.
The node $A$ is the source and $B$ is the sink.\\
We put an arc of capacity $1$ from $A$ to each of $a_i$, and an arc of infinite (or very large) capacity from $a_i$ to $b_j$ if card $j$ is in stack $i$.
There will be exactly four arcs leaving each $a_i$ and exactly one arc entering each $b_j$.\\
We also put an arc of infinite capacity from $b_j$ to $c_k$ if card $j$ has rank $k$, so again there will be exactly one arc leaving each of the nodes $b_j$ and exactly four arcs entering each of the nodes $c_k$.\\
Lastly, put an arc of capacity $1$ from each node $c_k$ to the sink $B$.
\begin{claim}
    The minimum capacity of any cut is $13$.
\end{claim}
\begin{proof}
    The cut $\{(A,a_i):i=1,\ldots,13\}$ has capacity $13$.
    We shall show that any other cut has capacity at least $13$.
    Consider a cut with finite capacity, then it does not contain any arc of the form $(a_i,b_j)$ or $(b_j,c_k)$.
    If there are $r$ arcs of the form $(c_k,B)$ for some $k$, then there are at least $s=13-r$ nodes $c_k$ still connected to $B$.
    Each of these $s$ many $c_k$'s is connected to four of the nodes $b_j$ exclusively, so $4s$ many $b_j$ is connected to $B$.
    But then at least $(4s)/4=s$ many nodes $a_i$ are connected to those $b_j$, hence to $B$, and they are also connected to $A$.
    So there are at least $s$ arcs of the form $(A,a_i)$ to be removed in order to complete the cut.
    This produces a total capacity of at least $r+s=13$.
\end{proof}
So by Theorem \ref{max-flow-min-cut}, there is a flow of value $13$.
But the capacities are either $1$ or infinity (or a very big integer), the Ford-Fulkerson algorithm starting with initial flow $0$ will terminate.
Hence there is an optimal flow (whose flows along every arc must be $0$ or $1$ by assumption).
Let $x$ be the optimal flow, then necessarily $x_{A,a_i}=1$.
As it is a flow, there is some $j$ such that $x_{a_i,b_j}=1$ and some $k$ such that $x_{b_j,c_k}=1$.
So we can match nodes in the way $a_i,c_k$ which provides the solution.
\subsection{The Transportation Problem}
The problem is as follows:
Imagine there are $m$ suppliers of a good labelled $1,\ldots,m$.
For any $i$, the supplier $i$ can supply $S_i$ units of the good.
There are $n$ destinations labelled $1,\ldots,n$ that require the supply of the good.
For any $j$, destination $j$ requires $D_j$ units of the good.
For each pair $(i,j)$, there is an associated unit transport cost $d_{i,j}$.
It is assumed in addition that the supply and demand matches, i.e.
$$\sum_{i=1}^mS_i=\sum_{j=1}^nD_j$$
This is basically the linear program to minimise the total cost $\sum_{i,j}d_{i,j}x_{i,j}$ subject to
$$\begin{cases}
    \forall i,\sum_{j=1}^nx_{i,j}=S_i\\
    \forall j,\sum_{i=1}^mx_{i,j}=D_j\\
    \forall i,j,x_{i,j}\ge 0
\end{cases}$$
The Lagrangian is then
\begin{align*}
    L(x,\lambda,\mu)&=\sum_{i,j}d_{i,j}x_{i,j}+\sum_{i=1}^m\lambda_i\left( S_i-\sum_{j=1}^n x_{i,j} \right)+\sum_{j=1}^n\mu_j\left( D_j-\sum_{i=1}^mx_{i,j} \right)\\
    &=\sum_{i,j}(d_{i,j}-\lambda_i-\mu_j)x_{i,j}+\sum_{i=1}^m\lambda_iS_i+\sum_{j=1}^n\mu_jD_j
\end{align*}
Note that $x_{i,j}\ge 0$, so we have $d_{i,j}\ge \lambda_i+\mu_j$ for the Lagrange multipliers to be feasible, hence
$$\Lambda=\{(\lambda,\mu):\inf_{x\ge 0}L(x,\lambda,\mu)>-\infty\}=\{(\lambda,\mu):\forall i,j,d_{i,j}\ge \lambda_i+\mu_j\}$$
If $(x_{i,j})$ is optimal for the problem, then there are feasible $(\lambda,\mu)$ such that
$$\forall i,j,(d_{i,j}-\lambda_i-\mu_j)x_{i,j}=0$$
by complementary slackness.
We can of course use simplex algorithm as it is a linear program, but the additional structure here allows us to simplify it.
\subsection{The Transportation Algorithm}
We design an algorithm for it as follows:
First, find an initial feasible assignment.
As there are $m+n-1$ linearly independent relations constraining the feasible set, any basic feasible solution will have $m+n-1$ basic variables.
There are two ways to do this: the North-West method and the greedy algorithm.
These will be introduced in a second.\\
After the initial feasible assignment is done, we assign Lagrange multipliers.
We may take $\lambda_1=0$ and enforce complementary slackness by choosing $\lambda_i$ and $\mu_i$ subject to $d_{i,j}=\lambda_i+\mu_j$ for each basic cell.\\
Next, we check for optimality by looking at the dual feasibility of the Lagrange multipliers.
The feasibility constraint is just $d_{i,j}\ge\lambda_i+\mu_j$ for each $i,j$.
If this is valid, then we are done.\\
Otherwise, we do the pivoting operation by picking one of the cells $(i,j)$ such that $\lambda_i+\mu_j>d_{i,j}$.
the rule is to pick the cell with the largest difference $\lambda_i+\mu_j-d_{i,j}$.
Now put an amount $\epsilon>0$ units of flow into the pivot cell.
At the same time add or substract $\epsilon$ from the basic cell to maintain feasibility.
We will try to choose the largest $\epsilon$ possible such that the flow is feasible.
Then do the same thing again and again until we reach optimality.
\begin{example}
    Suppose the table of supply, demand and cost is
    $$\begin{array}{c|c|c|c|c|c}
        d_{i,j}&&&&&D_j\\
        \hline
        &8&6&7&5&12\\ \hline
        &4&3&5&4&8\\ \hline
        &9&8&6&7&11\\ \hline
        S_i&7&6&10&8&
    \end{array}$$
    The North-West method is filling in the number of unit in each transport flow starting from North-West (top-left) of the array and fit in as many as we can, and the precedence of filling is from top-left to bottom-right.
    This way would give the basic feasible solution:
    $$\begin{pmatrix}
        7&5&&\\
        &1&7&\\
        &&3&8
    \end{pmatrix}$$
    Then we assign Lagrange multipliers by complementary slackness:
    $$\begin{array}{c|cccc|c}
        &8&6&8&9&\mu_j\\ \hline
        0&7&5&0&0&\\
        -3&0&1&7&0&\\
        -2&0&0&3&8&\\ \hline
        \lambda_i&&&&&
    \end{array}$$
    Now we test the optimality by writing out the payoffs, i.e. $\lambda_i+\mu_j-d_{i,j}$ in each of the zero cells (denoted by the terms in square brackets), which gives
    $$\begin{array}{c|cccc|c}
        &8&6&8&9&\mu_j\\ \hline
        0&7&5&[1]&[4]&\\
        -3&[1]&1&7&[2]&\\
        -2&[-3]&[-4]&3&8&\\ \hline
        \lambda_i&&&&&
    \end{array}$$
    So we need to pivot the top right corner, i.e. make it basic.
    Suppose it is changed to $\epsilon$, then we need to change our solution to
    $$\begin{pmatrix}
        7&5-\epsilon&&\epsilon\\
        &1+\epsilon&7-\epsilon&\\
        &&3+\epsilon&8-\epsilon
    \end{pmatrix}$$
    which is always feasible.
    For the solution to be basic, we take $\epsilon=5$ to swipe out an entry to get the new solution, which, after calculating Lagrange multipliers and payoffs,
    $$\begin{array}{c|cccc|c}
        &8&2&4&5&\mu_j\\ \hline
        0&7&[-4]&[-3]&5&\\
        1&[5]&6&2&[2]&\\
        2&[1]&[-4]&8&3&\\ \hline
        \lambda_i&&&&&
    \end{array}$$
    So we pivot the second row of the first column with, as we can find, has $\epsilon=2$.
    Doing this again and again and again and again and again and again and again and again and again yields our final table
    $$\begin{array}{c|cccc|c}
        &7&6&4&5&\mu_j\\ \hline
        0&[-1]&5&[-3]&7&\\
        -3&7&1&[-4]&[-2]&\\
        2&[0]&[0]&10&1&\\ \hline
        \lambda_i&&&&&
    \end{array}$$
    where all the entries in the square bracket is nonpositive, hence this is the optimal solution.\\
    The greedy algorithm works as well.
    This time we start with the cell of lowest cost and fill in as many as possible, then the second lowest, etc..
    In our case, this will produce the initial feasible assignment
    $$\begin{pmatrix}
        6&&&6\\
        &6&&2\\
        1&&10&
    \end{pmatrix}$$
    to which we can apply the same method to get to optimality.
\end{example}
\begin{remark}
    If the original posture of the prblem is such that the total supply exceeds the total demand, then we can reformulate the problem by adding another destination with demand
    $$D_{n+1}=\sum_{i=1}^mS_i-\sum_{j=1}^nD_j$$
    and set $d_{i,n+1}=0$ for all $i$ to ensure that all differences are consumed.
\end{remark}
Why does pivoting decrease the total cost?
For any feasible $x$ and any collection $(\lambda_i)$ and $(\mu_j)$ of Lagrange multipliers we have
\begin{align*}
    f(x)&=L(x,\lambda,\mu)\\
    &=\sum_{i,j}(d_{i,j}-\lambda_i-\mu_j)x_{i,j}+\sum_{i=1}^m\lambda_iS_i+\sum_{j=1}^n\mu_jD_j
\end{align*}
So suppose $x_0$ is the initial feasible assignment, then we can choose $\lambda_{i,0}$ and $\mu_{i,0}$ to enforce complementary slackness $(d_{i,j,0}-\lambda_{i,0}-\mu_{j,0})x_{i,j,0}=0$, then
$$f(x_0)=\sum_{i=1}^m\lambda_{i,0}S_i+\sum_{j=1}^n\mu_{j,0}D_j$$
Let $B=\{(i,j):x_{i,j,0}>0\}$ and $N=\{(i,j):x_{i,j,0}=0\}$ be the basis of the basic feasible solution $x_0$.
We choose $\lambda_{i,0}$ and $\mu_{i,0}$ such that $\lambda_{i,0}+\mu_{j,0}=d_{i,j}$ for all $(i,j)\in B$.
Suppose the next feasible assignment $x_1$ is non-degenerate, as one of the initial non-basic cells (the pivot cell) is made basic, there is a cell $(i_N,j_N)\in N$ such that
$$x_{i_N,j_N,0}=0,x_{i_N,j_N,1}=\epsilon>0$$
Then we obtain
\begin{align*}
    f(x_1)&=\sum_{(i,j)\in B}(d_{i,j}-\lambda_{i,0}-\mu_{j,0})x_{i,j,1}+\sum_{(i,j)\in N}(d_{i,j}-\lambda_{i,0}-\mu_{j,0})x_{i,j,1}
    +f(x_0)\\
    &=(d_{i_N,j_N}-\lambda_{i_N,0}-\mu_{j_N,0})\epsilon+f(x_0)\\
    &>f(x_0)
\end{align*}
by assumption.