\section{Linear Programming}
Recall that a linear program is the maximisation of $c^\top x$ subject to $Ax\le b,x\ge 0$, where $c\in\mathbb R^n$, $A$ is an $m\times n$ matrix and $b\in\mathbb R^m$.
We also know its dual problem in the form of minimising $b^\top\lambda$ subject to $A^\top\lambda\ge c,\lambda\ge 0$.
\subsection{The Fundamental Theorem}
\begin{theorem}[Fundamental Theorem of Linear Programming]
    Consider the primal problem stated.
    A vector $x^\star\in\mathbb R^n$ is optimal for the primal problem iff there is some vector $\lambda^\star\in\mathbb R^m$ such that:\\
    1. $x^\star$ is feasible for the primal problem.\\
    2. $\lambda^\star$ is feasible for the dual problem.\\
    3. $(b-Ax^\star)^\top\lambda^\star=0=(c-A^\top\lambda^\star)^\top x^\star$ (i.e. the complementary slackness).\\
    Also, if these are true, then $\lambda^\star$ is optimal for the dual problem, and $c^\top x^\star=b^\top\lambda^\star$.
\end{theorem}
\begin{proof}
    For the `if' direction, note that if $x,\lambda$ are feasible for the primal and dual problems, then by Theorem \ref{weak_dual}, $c^\top x\le b^\top\lambda$.
    But if $x^\star$ and $\lambda^\star$ satisfy complementary slackness, $c^\top x^\star=b^\top\lambda^\star$, consequently $c^\top x\le b^\top\lambda^\star\le c^\top x^\star$ for all feasible $x$ and $b^\top\lambda\ge c^\top x^\star=b^\top\lambda^\star$ for all feasible $\lambda$.
    Hence $x^\star$ and $\lambda^\star$ are optimal for their respective problems.\\
    For the ``only if'' direction, note that the objective function $f(x)=c^\top x$ and constraint function $g(x)=Ax$ are linear, hence are both concave and convex at the same time.
    In particular, $f$ is concave while $g$ is convex.
    Hence the value function of the primal function is concave, therefore Theorem \ref{lagrange_nece} guarantees the existence of $\lambda^\star$.
\end{proof}
\subsection{Extreme Points}
Consider the maximisation of $\psi(x)$ subject to $x\in X$ where $X$ is convex and $\psi$ is also convex.
In particular, for any $p\in(0,1)$,
$$\psi(px+(1-p)y)\le p\psi(x)+(1-p)\psi(y)\le\max\{\psi(x),\psi(y)\}$$
Hence the maximum of $\psi$ on any segment occurs at one of the endpoints.
Therefore it is sufficient to consider the maximisation problem in the set of points in $X$ that is not contained properly in a line segment.
\begin{definition}
    Let $X\subset\mathbb R^n$ be a convex set.
    A point $x\in X$ is an extreme point if the only case where there can be $y,z\in X,p\in (0,1)$ with $x=py+(1-p)z$ is when $x=y=z$.
\end{definition}
Extreme points are of course very useful in linear programs as linear functions are both concave and convex.
The concavity tells us we can characterise the optimal solution via duality.
As the set of feasible solutions is convex, the convexity tells us we only need to search for optimal solutions to the extreme points of the set of feasible solutions.
In particular, if the linear program has an optimal solution, then it has a optimiser that is an extreme point.
\subsection{Basic Feasible Solutions}
As we have seen, it is helpful if we formulate linear programs in the form of inequality functional constraints.
There is another form which we can write the linear programs in order to make it more convenient to optimise.
\begin{definition}
    A linear program is in standard form if the constraint can be written as $Ax=b,x\ge 0$.
\end{definition}
All linear programs can be put into the standard form.
For instance, $Ax\le b,x\ge 0$ can be transformed to $Ax+z=b,x,z\ge 0$ by our previous idea of slack variables.
Simply $Ax=b$ can be modified to $A(x-y)=b,x,y\ge 0$.
Combining the ideas, $Ax\le b$ can be transformed to $A(x-y)+z,x,y,z\ge 0$.
\begin{definition}
    Given an $m\times n$ matrix $A$ with $n>m$ and $b\in\mathbb R^m$.
    A solution $x\in\mathbb R^n$ of the equation $Ax=b$ is called basic if at most $m$ entries of $x$ are nonzero, i.e. $x_i\neq 0$ holds for at most $m$ different $i\in\{1,\ldots,n\}$.\\
    If $x$ is a basic solution and $x\ge 0$, it is called a basic feasible solution.
\end{definition}
Now denote the set of feasible solutions by $C=\{x\ge 0:Ax=b\}$, then we have
\begin{theorem}
    If $x$ is an extreme point of $C$, then $x$ is a basic feasible solution.
\end{theorem}
\begin{proof}
    We will prove the contrapositive.
    Let $x$ be a point in $C$ that is not basic, then there are at least $m+1$ indices $i_1,\ldots,i_r$ with $x_{i_k}>0$.
    Then the set $\{A_{i_1},\dots,A_{i_r}\}$ of columns of the matrix $A$ would be linearly dependent as $t\ge m+1$, therefore there exists $(w_1,\ldots,w_r)\neq 0$ such that $w_aA_{i_a}=0$ (summation implied).
    We construct
    $$z_i=\begin{cases}
        w_k\text{, if $i=i_k$ for some $k=1,\ldots,r$}\\
        0\text{, otherwise}
    \end{cases}$$
    Then $Az=0$ and $z\neq 0$, so $A(x\pm\epsilon z)=Ax=b$ for any $\epsilon$ for any $\epsilon$.
    As $x_{i_k}>0$ for any $k$, it is possible to choose some $\epsilon>0$ such that $x\pm\epsilon z\ge 0$.
    So $x\pm\epsilon z\in C$ and
    $$x=\frac{1}{2}(x+\epsilon z)+\frac{1}{2}(x-\epsilon_z)$$
    Hence $x$ is not extreme.
\end{proof}
This theorem has a partial converse.
\begin{theorem}
    Suppose that every set of $m$ columns of $A$ is linearly independent and $x$ is a basic feasible solution, then $x$ is necessarily an extreme point of $C$.
\end{theorem}
\begin{proof}
    By definition, at most $m$ many indices $i$ satisfy $x_i>0$.
    Suppose there is some $y,z\in C$ and $p\in (0,1)$ such that $x=py+(1-p)z$, then $y,z\ge 0$, so whenever $x_i=0$ we have $y_i=z_i=0$.
    Thus there are at least $n-m$ indices $i$ satisfying $y_i=z_i=0$.
    But also $Ay=b=Az$, so
    $$0=A(y-z)=\sum_{i}w_iA_i$$
    where $w=y-z$.
    But at most $m$ entries of $w$ are nonzero while any set of $m$ columns of $A$ is linearly independent, whence we conclude $w=0$.
    It follows that $x=y=z$.
\end{proof}
In conclusion, we can find the optimal solution of a linear program by checking each of the basic feasible solutions, whcih are easier to find and, more importantly, finite.
\subsection{Algorithms for Linear Programs}
Consider a linear program in standard form of maximising $c^\top x$ subject to $Ax=b,x\ge 0$ where $c\in\mathbb R^n,b\in\mathbb R^m$ and $A$ is an $m\times n$ matrix.
Assume that $n>m$ and every set of $m$ columns of $A$ is linearly independent, and that the problem is non-degenerate, i.e. every basic feasible solution $x$ has exactly $m$ indices $i$ with $x_i\neq 0$.
Our goal is to find a mechanical method of finding the optimal solution to the problem.\\
A na\"ive method to do this is to check all extreme points of $C$.
But we also know that extreme points are precisely the basic feasible solutions.
Thus we can roll out an algorithm as follows:\\
First, fix $B\subset\{1,\ldots,n\}$ with $|B|=m$ write and $N=\{1,\ldots,n\}\setminus B$.
Such a $B$ is called a basis of the solution.
Suppose $B=\{i_1,\ldots,i_m\}$, then we define $A_B=(A_{i_1}\ldots A_{i_m})$ which is an invertible matrix by assumption.
For any $x\in\mathbb R^n$, denote $x_B=(x_{i_1},\ldots,x_{i_m})^\top$.
The quantities $c_B,A_N,x_N,c_N$ are defined similarly.\\
Our aim is to find a basic solution of $Ax=b$.
We can write the problem instead in the form $A_Bx_B+A_Nx_N=b$.
Set $x_N=0$ shall yield $x_B=A_B^{-1}b$ as $A_B$ is invertible.
By a rearrangement of coordinate we can write the solution as
$$x=\begin{pmatrix}
    x_B\\
    x_N
\end{pmatrix}=\begin{pmatrix}
    A_B^{-1}b\\
    0
\end{pmatrix}$$
which is basic.
Now $x$ is feasible iff $A_B^{-1}b\ge0$ (which can be checked).
Assuming $x$ is a basic feasible solution, then we can just compute the objective function $c^\top x=c_B^\top x_B$.
Repeat this procedure for all possible choices of $B$ gives a list of all basic feasible solutions and the value of objective function evaluated there.
This is a finite set, so we can find its maximiser which is the solution to the problem.\\
There are some issues with this approach.
Firstly, if $n$ is large, the number of choices of $B$, that is $\binom{n}{m}$, would grow very fast, making it more computationally complex.
Also, this method assumed the existence of maximiser which is not always the case.
Our idea algorithm should certainly detect it when a maximiser does not exist, i.e. the problem is unbounded.\\
There is a better method to accomplish what we wanted.
We also use the idea of computing the basic feasible solutions like above, but we will try not to evaluate all candidate solutions.
By calculation we can show that the dual problem is to minimise $b^\top\lambda$ subject to $A^\top\lambda\ge c$.
So by the fundamental theorem, a feasible $x^\star$ is optimal for our primal problem iff there is a feasible (in the dual problem) Lagrange multiplier $\lambda^\star\in\mathbb R^m$ such that complementary slackness $(c-A^\top\lambda^\star)^\top x^\star=0$ is satisfied.\\
Consider a basic feasible solution $x=(A_B^{-1}b,0)^\top$, then if it is optimal, then there is some $\lambda$ satisfying complementary slackness
$$0=(c-A^\top\lambda)^\top x=(c_B-A_B^\top\lambda)^\top x_B$$
But no index of $x_B$ is zero by assumption, hence $\lambda=(A_B^\top)^{-1}c_B$.
So we can roll out our algorithm in the same way as before, but each time we check as well the feasibility of $\lambda$.
We can stop if such $\lambda$ is feasible, because by the fundamental theorem we have already found the optimal solution.
We also know when the problem is unbounded, that is when none of the basic feasible solution gives a feasible $\lambda$.
\subsection{The Simplex Algorithm}
A third way, which is known as the simplex algorithm, is a yet faster method.
So far our algorithms depend on iterating all candidates for basic feasible solutions randomly, which inspires us to find the solutions in a particular order to speed up the process.
In particular, if our current basic feasible solution does not work, we will want to find the next basic feasible solution that increases the objective function.\\
We start with an initial basic feasible solution $x_0$ whose components with indices in $B_0\subset\{1,\ldots,n\}$ are nonzero.
Write $N_0=\{1,\ldots,n\}\setminus B_0$.
We can associate a Lagrange multiplier $\lambda_0$ by complementary slackness, i.e. one that satisfies $(c-A^\top\lambda_0)_{B_0}=0$.
In particular, $(c-A^\top\lambda_0)^\top x_0=0$, so $c^\top x_0=b^\top\lambda_0$.
Let $\mu_0=c-A^\top\lambda_0$.
If $\mu_0\le 0$ then we are done and $x_0$ is optimal, otherwise we choose $x_1$ by the following:
\begin{claim}
    Suppose there is some $j\in N_0$ such that $\mu_{j,0}>0$ and there is an $i\in B_0$ such that there is a basic feasible solution $x_1$ with basis $B_1=B_0\cup\{j\}\setminus\{i\}$.
    Then
    $$c^\top x_1=c^\top x_0+\mu_{j,0}x_{j,1}>c^\top x_0$$
\end{claim}
\begin{proof}
    If $x$ is any feasible solution, then
    $$c^\top x=c^\top x+\lambda_0^\top(b-Ax)=b^\top\lambda_0+(c-A^\top\lambda_0)^\top x=c^\top x_0+\sum_{k\in N_0}\mu_{k,0}x_k$$
    as $\mu_{k,0}=0$ for any $k\in B_0$.
    The sum is then
    $$\sum_{k\in N_0}\mu_k x_{k,1}=\sum_{k\in N_1}\mu_kx_{k,1}+\mu_jx_{j,1}-\mu_ix_{i,1}=\mu_jx_{j,1}$$
    Also $x_{j,1}>0$ by non-degeneracy assumption.
    The claim follows.
\end{proof}
\begin{proposition}\label{simplex_unbounded}
    Suppose there is a $j\in N_0$ such that $\mu_{j,0}>0$ and for every $i\in B_0$ the basis solution with basis $B_1=B_0\cup\{j\}\setminus\{i\}$ is infeasible, then the problem is unbounded.
\end{proposition}
So if no feasible $x_1$ exists, then we can stop the algorithm and conclude that the problem is unbounded.
Otherwise, we can repeat the algorithm again and again, and stop either at a basic feasible solution whose accompanied Lagrange multiplier is feasible in the dual problem or at a point where we can conclude the unboundedness of the problem.
We may, however, have to go through all basic feasible solutions if we start with a bad guess, but at least we know after each step we shall obtain a better solution candidate.
\subsection{Simplex Algorithm in Action}
We are still considering the problem of maximising $c^\top x$ subject to $Ax=b,x\ge 0$, assuming the same things we assumed previously.
Suppose we know one basic feasible solution $x_0$, we want to use the simplex algorithm.
Before that, we need to do some pre-processing of the problem.
Let $B$ be the basis of $x_0$ and $N=\{1,\ldots,n\}\setminus B$ as usual.
For $x\in\mathbb R^n$, we rearrange the components $x=(x_B,x_N)^\top$ as usual.
Furthermore, the objective function can be written as
$$c^\top x=c^\top x_0+\mu_{N,0}^\top x_N$$
for feasible $x$.
where $\mu_0=c-A^\top\lambda_0$ and $\lambda_0=(A_B^\top)^{-1}c_B$ is the Lagrange multiplier obtained by complementary slackness.
The set of feasible solutions is then subject to the condition
$$x_B+A_B^{-1}A_Nx_N=x_{B,0},x_B,x_N\ge 0$$
To use the simplex algorithm, we first formulate the initial \text{simplex tableau} as the $(m+1)\times(n+1)$ matrix (in practice some of the columns are shuffled due to the position of the basis)
$$\Gamma=\begin{pmatrix}
    I&A_B^{-1}A_N&x_{B,0}\\
    0&\mu_{N,0}^\top&-c^\top x_0
\end{pmatrix}$$
where $I$ is the $m\times m$ identity.
We first test for optimality of $x_0$.
If $\mu_0\le 0$, then we can stop as it would be optimal due to the fundamental theorem.
Otherwise, we choose the pivot column, i.e. $j\in N$ such that $\mu_{j,0}>0$.
a rule of thumb is to pick $j$ such that $\mu_{j,0}$ is the largest.
We also pick the pivot row by looking within the pivot column $j$ and find the $i\in B$ which minimises $x_{i,0}/\Gamma_{i,j}$ over all $i\in B$ with $\Gamma_{i,j}>0$.
If this cannot be done, i.e. $\Gamma_{i,j}\le 0$ for all $i$, then we can stop as it indicates the unboundedness of the problem.\\
Assume we go through these steps without stopping the algorithm, we perform the pivot operation to move to the next basic feasible solution.
To do this, we scale row $i$ with $1/\Gamma_{i,j}$ and replace row $k$ with
$$(\text{old row }k)-(\text{old row }i)\times\frac{\Gamma_{k,j}}{\Gamma_{i,j}}$$
for all $k\neq i$ and run the algorithm on this new simplex tableau.
\begin{remark}
    1. For the initial basic feasible solution we have $x_{i,0}>0$ and $x_{j,0}=0$, whilst for the next basic feasible solution $x_{i,1}=0$ and $x_{j,1}=x_{i,0}/\Gamma_{i,j}>0$.\\
    2. The pivot operation is actually just a Gaussian elimination.\\
    3. The $\Gamma$ we obtained after the pivot operation is the same $\Gamma$ had we start with $x_1$ instead.\\
    4. Suppose $\Gamma_{i,j}\le 0$ for all $i\in B$, then for $i\in B$ and for $r>0$ let $x_r=x_0+r(\delta_j-\Gamma_{i,j}\delta_i)$ where $\delta_k$ is the $k^{th}$ standard basis.
    Then the transformation $x_0\mapsto x_r$ replaces $x_{i,0}$ with $x_{i,r}=x_{i,0}-r\Gamma_{i,j}$ and $x_{j,0}=0$ with $x_{j,r}=r$ and leaves other entries unchanged.
    Now $x_r$ is feasible since $x_r\ge 0$ and
    $$\begin{pmatrix}
        I&A_B^{-1}A_N
    \end{pmatrix}x_r=x_0$$
    But $c^\top x_r=c^\top x_0+r\mu_{j,0}\to\infty$ as $r\to\infty$, so the problem is unbounded.
    This proves Proposition \ref{simplex_unbounded}.
\end{remark}
\begin{example}
    Consider the linear program to maximise $3x_1+2x_2$ subject to
    $$\begin{cases}
        2x_1+x_2\le 4\\
        2x_1+3x_2\le 6\\
        x_1,x_2\ge 0
    \end{cases}$$
    Then its dual problem is to minimise $4\lambda_1+6\lambda_2$ subject to
    $$\begin{cases}
        2\lambda_1+2\lambda_2\ge 3\\
        \lambda_1+3\lambda_2\ge 2\\
        \lambda_1,\lambda_2\ge 0
    \end{cases}$$
    By introducing slack variables, the primal problem has constraint
    $$\begin{cases}
        2x_1+x_2+z_1=4\\
        2x_1+3x_2+z_2=6\\
        x_1,x_2,z_1,z_2\ge 0
    \end{cases}$$
    and the dual problem has constraint
    $$\begin{cases}
        2\lambda_1+2\lambda_2-v_1= 3\\
        \lambda_1+3\lambda_2-v_2= 2\\
        \lambda_1,\lambda_2,v_1,v_2\ge 0
    \end{cases}$$
    So the complementary slackness translates to
    $$x_1v_1=x_2v_2=\lambda_1z_1=\lambda_2z_2=0$$
    Running the simplex algorithm shall yield the maximiser $x_1=3/2,x_2=1$ with then corresponding dual solution $\lambda_1=5/4,\lambda_2=1/4$.
    We start with the simplex tableau (asterisked columns are where the identity matrix and the zero vectors are at)
    $$\begin{array}{c|cccc|c}
        &&&\ast&\ast&\\
        &x_1&x_2&z_1&z_2&\\ \hline
        z_1&2&1&1&0&4\\
        z_2&2&3&0&1&6\\ \hline
        \text{payoff}&3&2&0&0&0
    \end{array}$$
    Now the payoff row is not non-positive, hence we have not arrived at the optimal solution.
    As $3$ is the largest payoff entry, we let $x_1$ enter the basis, so the first column is the pivot column.
    We choose the first row to be the pivot row.
    The the pivot operation yields the next simplex tableau:
    $$\begin{array}{c|cccc|c}
        &\ast&&&\ast&\\
        &x_1&x_2&z_1&z_2&\\ \hline
        x_1&1&1/2&1/2&0&2\\
        z_2&0&2&-1&1&2\\ \hline
        \text{payoff}&0&1/2&-3/2&0&-6
    \end{array}$$
    It is still not optimal due to the positive $1/2$ entry in the payoff row.
    So this time the pivot column is the second column and we choose the second row as the pivot row.
    This gives us
    $$\begin{array}{c|cccc|c}
        &\ast&\ast&&&\\
        &x_1&x_2&z_1&z_2&\\ \hline
        x_1&1&0&3/4&-1/4&3/2\\
        x_2&0&1&-1/2&1/2&1\\ \hline
        \text{payoff}&0&0&-5/4&-1/4&-13/2
    \end{array}$$
    which we realise is optimal as the payoff is nonpositive.
\end{example}
\begin{remark}
    1. If the linear program is a minimisation instead of a maximisation problem, then the stopping criterion would be the tableau being nonnegative.\\
    2. When the problem is of the form like above, i.e. maximising $c^\top x$ subject to $Ax\le b,x\ge 0$, one observe that the coefficient of the payoff row under the slack variables are minus the corresponding dual variables (via complementary slackness).
    For the final tableau, the feasible dual variables are the Lagrange multipliers for the problems.\\
    3. In the example, we have
    $$\begin{pmatrix}
        3/4&-1/4\\
        -1/2&1/2
    \end{pmatrix}\begin{pmatrix}
        2&1&1&0&4\\
        2&3&0&1&6
    \end{pmatrix}=\begin{pmatrix}
        1&0&3/4&-1/4&3/2\\
        0&1&-1/2&1/2&1
    \end{pmatrix}$$
    Hence the perturbed problem of maximising $3x_1+2x_2$ subject to
    $$\begin{cases}
        2x_1+x_2\le 4+\epsilon_1\\
        2x_1+3x_2\le 6+\epsilon_2\\
        x_1,x_2\ge 0
    \end{cases}$$
    with $\epsilon_{1,2}$ small yields the final tableau
    $$\begin{array}{c|cccc|c}
        &\ast&\ast&&&\\
        &x_1&x_2&z_1&z_2&\\ \hline
        x_1&1&0&3/4&-1/4&3/2+3\epsilon_1/4-\epsilon_2/4\\
        x_2&0&1&-1/2&1/2&1-\epsilon_1/2+\epsilon_2/2\\ \hline
        \text{payoff}&0&0&-5/4&-1/4&-13/2-5\epsilon_1/4-\epsilon_2/4
    \end{array}$$
    Therefore it has maximiser at
    $$x_1=\frac{3}{2}+\frac{3}{4}\epsilon_1-\frac{1}{4}\epsilon_1,x_2=1-\frac{1}{2}\epsilon_1+\frac{1}{2}\epsilon_2, 3x_1+2x_2=\frac{13}{2}+\frac{5}{4}\epsilon_1+\frac{1}{4}\epsilon_2$$
    given that they are feasible (possible for small enough $\epsilon_{1,2}$).
\end{remark}
\subsection{The Two-Phase Algorithm}
Again our problem is to maximise $c^\top x$ subject to $Ax=b,x\ge 0$.
In order to roll out the simplex algorithm, we need at least one basic feasible solution.
It is certainly tempting if we can know a way that can yield one efficiently.\\
There are a lot of approaches, for example just pick a basis and computed $A_B^{-1}b$ (using the notations we introduced earlier).
This is certainly a sufficient algorithm, but this might not be very efficient.\\
Consider a new problem of minimising $e^\top y$ subject to $Ax+y=b,x,y\ge 0$ where $e=(1,\ldots,1)^\top$.
The variable $y$ here is called an artificial variable.
Now $(x,y)=(0,b)$ is a basic feasible solution for this problem, so we can apply the simplex algorithm which must terminate at some $(x,y)=(x_0,0)$ given that the original problem is feasible.
Then $x_0$ has to be basic feasible for the original problem.
So we can modify the algorithm by adding a first phase to compute one basic feasible solution first (called Phase I) and then apply the simplex algorithm to it (which is Phase II).
\begin{example}
    Consider the problem to maximise $x_1-3x_2+5x_3$ subject to
    $$\begin{cases}
        x_1+x_2+x_3\le 30\\
        -x_2+2x_3=20\\
        -x_1+2x_2+x_3\ge 40\\
        x_1,x_2,x_3\ge 0
    \end{cases}$$
    we introduce slack variables as usual to transform the constraint to
    $$\begin{cases}
        x_1+x_2+x_3+z_1=30\\
        -x_2+2x_3=20\\
        -x_1+2x_2+x_3-z_2=40\\
        x_1,x_2,x_3,z_1,z_2\ge 0
    \end{cases}$$
    Add in artificial variable $y_1,y_2$,
    $$\begin{cases}
        x_1+x_2+x_3+z_1=30\\
        -x_2+2x_3+y_1=20\\
        -x_1+2x_2+x_3-z_2+y_2=40\\
        x_1,x_2,x_3,z_1,z_2,y_1,y_2\ge 0
    \end{cases}$$
    In some sense we should have added a variable $y_3$ on the first expression as well, but it is actually redundant as we already have an nonnegative expression.
    We solve this new problem with the simplex algorithm, with the initial tableau:
    $$\begin{array}{c|ccccccc|c}
        &&&&\ast&&\ast&\ast&\\
        &x_1&x_2&x_3&z_1&z_2&y_1&y_2&\\ \hline
        z_1&1&1&1&1&0&0&0&30\\
        y_1&0&-1&2&0&0&1&0&20\\
        y_2&-1&2&1&0&-1&0&1&40\\ \hline
        \text{Phase II}&1&-3&5&0&0&0&0&0\\ \hline
        \text{Phase I}&0&0&0&0&0&1&1&0
    \end{array}$$
    We have put the objective functions of both phases so that we can manipulate both simultaneously so that when Phase I is completed, we can immediately go to Phase II.\\
    Now the identity part of the upper part of the tableau does not correspond to the zero part of the Phase I payoff, so we do a transformation to modify the tableau:
    $$\begin{array}{c|ccccccc|c}
        &&&&\ast&&\ast&\ast&\\
        &x_1&x_2&x_3&z_1&z_2&y_1&y_2&\\ \hline
        z_1&1&1&1&1&0&0&0&30\\
        y_1&0&-1&2&0&0&1&0&20\\
        y_2&-1&2&1&0&-1&0&1&40\\ \hline
        \text{Phase II}&1&-3&5&0&0&0&0&0\\ \hline
        \text{Phase I}&1&-1&-3&0&1&0&0&-60
    \end{array}$$
    Note that as we are encountering a minimisation problem in Phase I, our aim is to make the last row nonnegative.
    Pivot once:
    $$\begin{array}{c|ccccccc|c}
        &&&\ast&\ast&&&\ast&\\
        &x_1&x_2&x_3&z_1&z_2&y_1&y_2&\\ \hline
        z_1&1&3/2&0&1&0&-1/2&0&20\\
        x_3&0&-1/2&1&0&0&1/2&0&10\\
        y_2&-1&5/2&0&0&-1&-1/2&1&30\\ \hline
        \text{Phase II}&1&-1/2&0&0&0&-5/2&0&-50\\ \hline
        \text{Phase I}&1&-5/2&0&0&1&3/2&0&-30
    \end{array}$$
    Pivot twice:
    $$\begin{array}{c|ccccccc|c}
        &&\ast&\ast&\ast&&&&\\
        &x_1&x_2&x_3&z_1&z_2&y_1&y_2&\\ \hline
        z_1&8/5&0&0&1&3/5&-1/5&-3/5&2\\
        x_3&-1/5&0&1&0&-1/5&2/5&1/5&16\\
        x_2&-2/5&1&0&0&-2/5&-1/5&2/5&12\\ \hline
        \text{Phase II}&4/5&0&0&0&-1/5&-13/5&1/5&-44\\ \hline
        \text{Phase I}&0&0&0&0&0&1&1&0
    \end{array}$$
    The last row is nonnegative and the objective function attains zero, so we have already achieved our aim for Phase I with the basic feasible solution shown in the tableau.
    This produces our initial tableau for Phase II:
    $$\begin{array}{c|ccccc|c}
        &&\ast&\ast&\ast&&\\
        &x_1&x_2&x_3&z_1&z_2&\\ \hline
        z_1&8/5&0&0&1&3/5&2\\
        x_3&-1/5&0&1&0&-1/5&16\\
        x_2&-2/5&1&0&0&-2/5&12\\ \hline
        \text{Phase II}&4/5&0&0&0&-1/5&-44\\
    \end{array}$$
    Now we are maximising, so we need to make the payoff nonpositive.
    After the first pivoting we have
    $$\begin{array}{c|ccccc|c}
        &\ast&\ast&\ast&&&\\
        &x_1&x_2&x_3&z_1&z_2&\\ \hline
        x_1&1&0&0&5/8&3/8&5/4\\
        x_3&0&0&1&1/8&-1/8&65/4\\
        x_2&0&1&0&1/4&-1/4&25/2\\ \hline
        \text{Phase II}&0&0&0&-1/2&-1/2&-45\\
    \end{array}$$
    which has nonpositive payoff, hence the optimal solution is at $(x_1,x_2,x_3)=(5/4,25/2,65/4)$.
\end{example}